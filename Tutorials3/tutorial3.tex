\documentclass{report}

\usepackage{amsmath,enumitem}

\DeclareMathOperator{\rank}{Rank\;of}

\begin{document}

\chapter{Introduction}
 In this tutotrial we see how to enumerate the items in Chapter \ref{ch:enumeration}, how to escape the space in math mode in Chapter \ref{ch:escapeSpaces} \dots \ldots  
 
 \begin{equation}
 	A = \begin{pmatrix}
 	  1 & 2 & \cdots & 10  \\
 	  4 & 5 &  \cdots & 14\\
 	  \vdots  & \vdots & \ddots & \vdots  \\
 	  10 & 11 & \cdots & 20 	  
 	\end{pmatrix}_{10 \times 10}
 \end{equation}

\begin{equation}
	a_1, \dots, a_n
\end{equation}

\begin{equation}
	a_1 \dots a_n
	\end{equation}



\chapter{Recap: How to enumerate items}\label{ch:enumeration}

\begin{enumerate}[label={(\alph*)}, start = 5]
	\item This is my first enumeration\dots 
	\item This is my second enumeration \dots 
	\item This is newly added item
\end{enumerate}

dsjfdkjlkf
dfjdklf
djfdkljfk
dfdkfjld
dnfkd

dfkdj

djfdkjf


\begin{enumerate}[label={(\alph*)}, resume]
	\item \label{it:1} This item has to be enumerate as a continuation of the previous 
	\item This is my second enumeration \dots 
\end{enumerate}

\ref{it:1}

\chapter{How to escape the spaces inside the math equations}\label{ch:escapeSpaces}
The spacing delims used in this tutorial will also work in text mode. 
\begin{equation}
f (x) = x \text{ and } g(x) = x^2
\end{equation}

\begin{equation}
f (x) = x \ \text{and}\ g(x) = x^2
\end{equation}


\begin{equation}
f (x) = x \, \text{and}\, g(x) = x^2
\end{equation}

\begin{equation}
f (x) = x \; \text{and}\; g(x) = x^2
\end{equation}

David\;Raj\;Micheal 

\begin{equation}
f (x) = x, \quad \quad \quad  g(x) = x^2
\end{equation}

\begin{equation}
f (x) = x, \qquad  g(x) = x^2
\end{equation}


f(x)=x $f(x)=x$ $\sin x$ $\rank A$

\chapter{How to insert tables}

\begin{table}[htb!]
	\centering 
	\begin{tabular}{|c|r|}
		\hline
		Name & Address \\
		\hline 
		Here & we learn \\
		how to & insert \\
		a table & \dots \\
		\hline 
	\end{tabular}
\caption{This is my first table}
\label{tab:tutorial}
\end{table}


The Table \ref{tab:tutorial} gives you \dots



\end{document}